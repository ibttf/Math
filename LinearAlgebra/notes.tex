\documentclass[12pt]{article}

\usepackage[utf8]{inputenc}
\usepackage{latexsym,amsfonts,amssymb,amsthm,amsmath}
\usepackage{geometry}
\usepackage{tcolorbox}  % For boxed environments
\usepackage{xcolor}     % For text colors
\usepackage[dvipsnames]{xcolor}
\geometry{margin=1in}

\setlength{\parindent}{0pt}
\setlength{\parskip}{1em}

% Custom command for solutions
\newcommand{\solution}{\noindent\textbf{Solution:}\par\nopagebreak}

% Custom boxed environment for definitions
\newtcolorbox{definitionbox}[1][]{
    colback=blue!5!white,
    colframe=blue!75!black,
    fonttitle=\bfseries,
    title=Definition,
    #1
}

% Custom boxed environment for key concepts in solutions
\newtcolorbox{keyconceptbox}[1][]{
    colback=red!5!white,
    colframe=red!75!black,
    fonttitle=\bfseries,
    title=Key Concept,
    #1
}

% Custom boxed environment for a black note
\newtcolorbox{note}[1][]{colback=white, colframe=black, title=Note, fonttitle=\bfseries, #1}




% Custom commands for difficulty levels with practice quetsions
\newcommand{\easy}[1]{\noindent\textcolor{OliveGreen}{\textbf{EASY:}} #1 \par}
\newcommand{\medium}[1]{\noindent\textcolor{orange}{\textbf{MEDIUM:}} #1 \par}
\newcommand{\hard}[1]{\noindent\textcolor{red}{\textbf{HARD:}} #1 \par}


% Custom command for a definition which will be bolded
\newcommand{\definition}[2]{
  \noindent\textbf{#1:} #2
}

\title{Linear Algebra \\ MATH 1201}
\author{Roy Lee}
\date{}  % No date

\begin{document}

\maketitle

\section{Linear Equations in Linear Algebra}

\subsection*{1.1 Systems of Linear Equations}

\definition{Linear equation}{An equation of the form $a_1x_1 + a_2x_2 + \cdots + a_nx_n = b$ where $a_1, a_2, \ldots, a_n, b$ are constants and $x_1, x_2, \ldots, x_n$ are variables.}

\definition{Solution set}{All possible solutions to a system of linear equations.}

\begin{definitionbox}
    \textbf{Types of solutions:} There are only three types of solutions a system of linear equations can have.
    \begin{itemize}
        \item No solution
        \item Exactly one solution
        \item Infinitely many solutions
    \end{itemize}
\end{definitionbox}



If we have a system of linear equations that looks like:
\begin{aligned}
\[
    x_1 & - 2x_2 & + x_3 & = 0 \\
     & x_2 & + 2x_3 & = 3 \\
    3x_1 & + x_2 & + 3x_3 & = 3

\]
\end{aligned}
\definition{Coefficient matrix}{Way of writing a system of linear equations}











\end{document}
